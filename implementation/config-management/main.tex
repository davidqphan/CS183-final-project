\subsection{Configuration Management}
Configuration Management for Acme Corp. is maintained using \lstinline$Puppet$. 
This setup allows new workstations to be deployed on the network automatically
with minor manual initialization (as described in \ref{puppet-client}).

\subsubsection{Network/Interface Information}
\textbf{Network Gateway} \\
The Puppet application is hosted on the backups server and as such, inherits 
the same network configuration described in section \ref{backups}. \\

\noindent
\textbf{IP Address} \\
See \textbf{Network Gateway} subsection above.

\subsubsection{Server Setup/Configuration}
The backups server is configured as the \textit{Puppet Master} and contains 
the config modules for the Puppet service. \\

\noindent
The Puppet modules configure all the applicable software and services on each 
host when a new host is configured on the network. Puppet runs periodically to 
check for new devices on the network with a \textit{Puppet Agent}. When a 
Puppet Agent is found on the network on a host that has not been configured,
Puppet securely logs in and configures the machine. \\

\subsubsection{Client Setup/Configuration} \label{puppet-client}
All new clients (host) on the network must be manually setup with the network 
gateway that will provide access to the network (IP of physical Fa switch 
port, in address space \textbf{10.1.10.0/24}). \\

\noindent
Additionally, Puppet (\textit{as Puppet Agent}) must be installed on each 
client. Once these initial configuration are done, the Puppet Master will 
automatically find the host and configure it to serve as a workstation for a 
new user. \\
