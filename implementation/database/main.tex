%%%%%%%%%%%%%%%%%%%%%%%%%%%%%%%%%%%%%%%%%%% start of database writeup %%%%%%%%%%%%%%%%%%%%%%%%%%%%%%%%%%%%%%%%%%%%%%%%%% 

\subsection{Database}
We are going with a MariaDB database for ACME as it widely used, easily 
maintainable, and secure. In order to set up the MariaDB database we would
need to install it both the client and server packages.

\begin{itemize}
\item sudo apt-get install mariadb-server
\end{itemize}

\noindent After running the installation of MariaDB we then would need to set 
up our admin user so we are then able to populate the database with data that 
ACME CORP needs to be stored. \\

\noindent By running:

\begin{itemize}
\item sudo mysql\_secure\_installation \\
\end{itemize}

\vspace{-1em}

\noindent We are able to do all of the following: 

\begin{itemize}
\item set root password
\item disable remote root login
\item remove test database
\item remove anonymous users and
\item reload privileges
\end{itemize}

\noindent After adding the data that needs to be stored we would then set up 
the user accounts and privileges for the HR department as they are the only 
department needing access to the database.

\begin{itemize}

\item MariaDB $>$ CREATE USER frankHR@'localhost' IDENTIFIED BY 'password'

\item MariaDB $>$ GRANT ALL PRIVILEGES on employees.* to frankHR@'\%';

\item MariaDB $>$ FLUSH PRIVILEGES; 

\end{itemize}

\noindent By the above commands we created a user account from Frank from HR 
and then have given Frank privileges to access the database 'employees'. 
Following this same style we would be able to add new tables and users and also
 give and take away privileges based on needs.

%%%%%%%%%%%%%%%%%%%%%%%%%%%%%%%%%%%%%%%%%%% end of database writeup %%%%%%%%%%%%%%%%%%%%%%%%%%%%%%%%%%%%%%%%%%%%%%%%%% 

