\subsection{Network File System}
The Network File System (NFS) service for Acme Corp. relies on the in-built 
support for NFS baked into the Linux kernel. 

\subsubsection{Network/Interface Information}
\textbf{Network Gateway} \\
The NFS server itself connects to the network through Fa0/5 of Switch 6 and is 
a member of the Acme Corp. "Server Zone". The default gateway for the NFS 
server is the static IP for Fa0/5 on the \textbf{10.1.10.0/24} subnet, TBD. \\

\noindent
\textbf{IP Address} \\
The NFS server is a member of the Internal Services ("Server Zone"), and is
likewise configured on subnet \textbf{10.1.160.x}, with a static IP address of 
\textbf{10.1.160.5}. \\

\noindent
\textbf{Assumption!}: The upgraded network requirements did not specify if Acme 
Corp. user directories are hosted from the NFS server. It is assumed that
business operations require departments to share files in the specific
department shared directory and that users will only store personal files in
their own home directories. With this assumption in mind, only
department-specific shares are configured on the NFS server.

\subsubsection{Server Setup/Configuration}
To enable the NFS functionality of the Linux kernel, the
\lstinline$nfs-kernel-server$ package must be installed on the server.

\begin{lstlisting}[backgroundcolor=\color{Gray}, language=bash]
 $ apt install nfs-kernel-server
\end{lstlisting}
\vspace{1em}

\noindent
Additionally, some config files must be modified from the defaults. \\

\begin{lstlisting}
 /etc/idmapd.conf
\end{lstlisting}
\vspace{1em}

\noindent
The only necessary change to this file is assigning the Acme Corp. domain. \\

\begin{lstlisting}[backgroundcolor=\color{Gray}]
 Domain = acme.com
\end{lstlisting}

\begin{lstlisting}
 /etc/exports
\end{lstlisting}

\noindent
The exports file defines which directories will be exported and to which
user groups. Per the upgraded network requirements, the NFS shares are assigned 
only to the respective subnets for which they are "allowed visibility". \\

\begin{lstlisting}[backgroundcolor=\color{Gray}]
 # mounts Executive share to IPs on the Executive VLAN
 /home/exec 10.1.110.0/24(rw)

 # mounts HR share to IPs on the Executive, and HR VLANs
 /home/hr 10.1.110.0/24(rw) 10.1.120.0/24(rw)

 # mounts R&D share to IPs on the Executive, HR, 
 # R&D, and Engineering VLANs
 /home/rnd 10.1.110.0/24(rw) 10.1.120.0/24(rw) \
     10.1.130.0/24(rw) 10.1.140.0/24(rw)

 # mounts Engineering share to IPs on the Executive, 
 # HR, R&D, and Engineering VLANs
 /home/engr 10.1.110.0/24(rw) 10.1.120.0/24(rw) \
     10.1.130.0/24(rw) 10.1.140.0/24(rw)

 # mounts Sales share to IPs on the Executive, HR, 
 # and Sales VLANs
 /home/sales 10.1.110.0/24(rw) 10.1.120.0/24(rw) \
     10.1.150.0/24(rw)
\end{lstlisting}

\subsubsection{Client Setup/Configuration}
To enable the NFS functionality on each client workstation, the
\lstinline$nfs-client$ package must be installed on each respective workstation.
\\

\begin{lstlisting}[backgroundcolor=\color{Gray}, language=bash]
 $ apt install nfs-client
\end{lstlisting}
\vspace{1em}

\noindent
Additionally, some config files must be modified from the defaults. \\

\begin{lstlisting}
 /etc/idmapd.conf
\end{lstlisting}
\vspace{1em}

\noindent
The only necessary change to this file is assigning NFS server domain (as
configured by the DNS). \\

\begin{lstlisting}[backgroundcolor=\color{Gray}]
 Domain = nfs.acme.com
\end{lstlisting}
\vspace{1em}

\noindent
Each client must be configured to mount the share on system boot. This can be
done by editing the following file on the client: \\
\begin{lstlisting}
 /etc/fstab
\end{lstlisting}
\vspace{1em}

\begin{lstlisting}[backgroundcolor=\color{Gray}]
 # [share] is a place holder for the appropriate 
 # department share folder
 nfs.acme.com:/home/[share]  /home/[share]   nfs defaults 0 0
\end{lstlisting}
* Some clients may require a line for each share that must be mounted. For
example, an Executive client must mount all network shares so the client 
configuration might look like: \\

\begin{lstlisting}[backgroundcolor=\color{Gray}]
 nfs.acme.com:/home/exec     /home/exec   nfs defaults 0 0
 nfs.acme.com:/home/hr       /home/hr     nfs defaults 0 0
 nfs.acme.com:/home/rnd      /home/rnd    nfs defaults 0 0
 nfs.acme.com:/home/engr     /home/engr   nfs defaults 0 0
 nfs.acme.com:/home/sales    /home/sales  nfs defaults 0 0
\end{lstlisting}
\vspace{1em}

\noindent
Additionally, each client must be configured with the \lstinline$autofs$ 
package. \\

\begin{lstlisting}[backgroundcolor=\color{Gray}, language=bash]
 $ apt install autofs
\end{lstlisting}
\vspace{1em}

\noindent
And the following files must be edited, respectively: \\

\begin{lstlisting}
 /etc/auto.master
\end{lstlisting}
\vspace{1em}

\begin{lstlisting}[backgroundcolor=\color{Gray}]
 /-  /etc/auto.mount
\end{lstlisting}

\begin{lstlisting}
 /etc/auto.mount
\end{lstlisting}
\vspace{1em}

\begin{lstlisting}[backgroundcolor=\color{Gray}]
 /home/[share] -fstype=nfs,rw nfs.acme.com:/home/[share]
\end{lstlisting}
* \lstinline$[share]$ is a place holder for each share that must be mounted, 
as before.
\vspace{1em}

\noindent
Finally, a directory should be created on each respective client for each NFS
share that will be mounted. \\
