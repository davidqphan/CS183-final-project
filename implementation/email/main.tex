\subsection{Email}

For our email setup, we are configuring a Postfix server to send email through
our external SMTP provider, SendGrid.

\subsubsection{Installing Postfix}

\begin{enumerate}
 \item { Install Postfix with the following command:
  \begin{lstlisting}[language=bash] 
  $ sudo apt-get install postfix
   \end{lstlisting}
 }
 \item { During installation, a prompt will appear asking for your
	 \textbf{General type of mail configuration.} Select \textbf{Internet
		Site.}
 }
 \item {Enter the fully qualified name of your domain, \textbf{fqdn.acmecorp.com}
 }
 \item {Once the installation is finished, edit the \textbf{main.cf} file with
	 your favorite text editor.
   \begin{lstlisting}[language=bash]
 $ sudo vim /etc/postfix/main.cf
   \end{lstlisting}
 }
 \item {Set \textbf{myhostname} parameter to be configured with your server's
	 FQDN:
   \begin{lstlisting}[language=bash]
 myhostname = fqdn.acmecorp.com
   \end{lstlisting}
 }
\end{enumerate}

\subsubsection{How to configure SMTP Usernames and Passwords}

\begin{enumerate}
\item { Open or create the \textbf{/etc/postfix/sasl\_passwd} file,
		using your favorite text editor: 
  \begin{lstlisting}[language=bash]
 $ sudo vim /etc/postfix/sasl_passwd
   \end{lstlisting}
 }
 \item { Add your destination (SMTP Host), username, and password in the
	 following format: 
  \begin{lstlisting}[language=bash]
 [smtp.sendgrid.net]:587 username:password 
   \end{lstlisting}
 }
 \item { Create the hash db file for Postfix by running the command:
   \begin{lstlisting}[language=bash]
 $ sudo postmap /etc/postfix/sasl_passwd 
   \end{lstlisting}
   If successful, there should be a new file generated, \textbf{sasl\_passwd.db}
		in the \textbf{/etc/postfix/} directory.
 }
\end{enumerate}

\subsubsection{How to secure your Password and Hash Database Files}

\begin{enumerate}
 \item { We change the permissions of \textbf{/etc/postfix/sasl\_passwd}
		and \textbf{/etc/postfix/sasl.passwd.db} so that only the
		\textbf\textit{root} user can read or write to the file. Run
		the commands:
  \begin{lstlisting}[language=bash]
 $ sudo chown root:root /etc/postfix/sasl_passwd 
 $ sudo chown root:root /etc/postfix/sasl_passwd.db
 $ sudo chmod 0600 /etc/postfix/sasl_passwd
 $ sudo chmod 0600 /etc/postfix/sasl_passwd.db
   \end{lstlisting}
 }
\end{enumerate}

\subsubsection{How to configure the Relay Server}

\begin{enumerate}
 \item { Open the \textbf{/etc/postfix/main.cf} file with your favorite
		text editor:
  \begin{lstlisting}[language=bash]
 $ sudo vim /etc/postfix/main.cf 
   \end{lstlisting}
 }
 \item { Update the \textbf{relayhost} parameter to show your external SMTP
	 relay host.
  \begin{lstlisting}[language=bash]
  relayhost = [smtp.sendgrid.net]:587 
   \end{lstlisting}
 }
 \item {At the end of the file, add the following parameters to enable
	 authentication:
 \begin{lstlisting}[language=bash]
 # enable SASL authentication
 smtp_sasl_auth_enable = yes
 # disallow methods that allow anonymous authentication.
 smtp_sasl_security_options = noanonymous
 # where to find sasl_passwd
 smtp_sasl_password_maps = hash:/etc/postfix/sasl_passwd
 # enable STARTTLS encryption
 smtp_use_tls = yes
 # where to find CA certificates
 smtp_tls_CAfile = /etc/ssl/certs/ca-certificates.crt
 \end{lstlisting}
 Save your changes.
}
 \item { Restart Postfix, by running the command:
 \begin{lstlisting}[language=bash]
 $ sudo service postfix restart
 \end{lstlisting}
 }
\end{enumerate}

\subsubsection{Design Justifications}
We connected our server to a mailing service, SendGrid because it helps
AcmeCorp avoid pitfalls like having our server IP getting blacklisted by
anti-spam services. We chose port 587 (mail submission agent) for our email client to email server
communication, because the MSA accepts email after authentication. It helps
stop outgoing spam when netmasters of DUL ranges can block outgoing connections
to the SMTP port (port 25).
