%%%%%%%%%%%%%%%%%%%%%%%%%%%%%%%%%%%%%%%%%%% start of AD writeup %%%%%%%%%%%%%%%%%%%%%%%%%%%%%%%%%%%%%%%%%%%%%%%%%% 

\subsection{Active Directory}
We chose to use OpenLDAP as our active directory protocol.

\begin{itemize}
	
	\item sudo apt-get install slapd ldap-utils
	
\end{itemize}

\vspace{.5em}

\noindent This will prompt you administrator password for the administrator LDAP 
account.
After changing some recommended setting we then enter:

\begin{itemize}
	\item dpkg-reconfigure slapd
\end{itemize}

\vspace{.5em}

\noindent The above command reconfigures with the updated information we entered. After 
running this command we are then asked for numerous pieces of information

\begin{itemize}
	\item DNS domain name
	\item Organization name (ACME)
	\item LDAP admin password which we created earlier
	\item Selection of backend database. 
\end{itemize}

\vspace{.5em}

\noindent After the steps above we then test the OpenLDAP by running:

\begin{itemize}
	\item ldapsearch -x
\end{itemize}

\vspace{.5em}

\noindent If the ‘Success’ message outputs, then Congratulations! Our LDAP 
Server is working!!\\ 

\noindent Now to install the LDAP Server Administration portion. Since we will 
have a team of users that might not be great with computers, we will go with
the GUI tool. Which will help the manage and configure the LDAP server.
We install it with the following command:

\begin{itemize}
	\item sudo apt-get install phpldapadmin
\end{itemize}

\vspace{.5em}

\noindent We then have to set symbolic link for the phpldapadmin directory:

\begin{itemize}
	\item ln -s /usr/share/phpldapadmin/ /var/www/html/phpldapadmin
\end{itemize}

\vspace{.5em}

\noindent We then need to edit the config.php file for setting correct time 
zone:

\begin{itemize}
	\item vim /etc/phpldapadmin/config.php
\end{itemize}

\vspace{.5em}

\noindent We will look for a line:
\begin{itemize}
	\item \begin{verbatim}
	$config->custom->appearance['timezone'] = ;
	\end{verbatim} 
\end{itemize} 

\vspace{.5em}

\noindent Change it to ACME Pennsylvania:
\begin{itemize}
	\item \begin{verbatim}
	$$config->custom->appearance['timezone'] = 'US/Pennsylvania';
	\end{verbatim}
\end{itemize}

\vspace{.5em}

\noindent Lastely we need to find and replace the domain names with our own. 
Find "Define LDAP Servers" section with in config file and 
change the following lines:
\begin{verbatim}
	// Set your LDAP server name //
	$servers->setValue('server','name','Unixmen LDAP Server');
	
	// Set your LDAP server IP address // 
	$servers->setValue('server','host','192.168.1.103');
	
	// Set Server domain name //
	$servers->setValue('server','base',array('dc=unixmen,dc=local'));
	
	// Set Server domain name again//
	$servers->setValue('login','bind_id','cn=admin,dc=unixmen,dc=local');
\end{verbatim}

\vspace{.5em}

\noindent We need to restart the apache service using:
\begin{itemize}
	\item systemctl restart apache2
\end{itemize}

\vspace{.5em}

\noindent Now make sure port "80" and port "389" are open in the 
firewall/router config and we are finished.

%%%%%%%%%%%%%%%%%%%%%%%%%%%%%%%%%%%%%%%%%%% end of AD writeup %%%%%%%%%%%%%%%%%%%%%%%%%%%%%%%%%%%%%%%%%%%%%%%%%% 
