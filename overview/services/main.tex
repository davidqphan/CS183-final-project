\subsection{Services}
All existing services must be upgraded to accommodate the network upgrade. 
The scope of each service upgrade will vary based on the need, but each
service will be reimplemented to better fit within the post-upgrade network 
architecture. 
\subsubsection{Existing Services}
The following existing services are upgraded (in no particular order):
\newcolumntype{P}[1]{>{\centering\arraybackslash}p{#1}}
\begin{center}
\begin{tabular}{|P{7.3cm}|P{5.5cm}|}
    \hline
    \textbf{Service Description} & \textbf{Preferred Package/Application} \\
    \hline
    Network File System (NFS) & nfs-kernel-server, nfs-client \\
    \hline
    Webserver & apache2 \\
    \hline
    Database & mariadb \\
    \hline
    Email & postfix \\
    \hline
    Active Directory (AD) & openldap \\
    \hline
    Domain Name Server (DNS) & \textcolor{red}{-\$30,000} \\
    \hline
    Dynamic Host Configuration Protocol (DHCP) & dhcpd \\
    \hline
\end{tabular}
\end{center}


\subsubsection{New Services}
The following new services are implemented on the new network (in no particular
order):
\newcolumntype{P}[1]{>{\centering\arraybackslash}p{#1}}
\begin{center}
\begin{tabular}{|P{7.3cm}|P{5.5cm}|}
    \hline
    \textbf{Service Description} & \textbf{Preferred Package/Application} \\
    \hline
    Virtual Local Area Network (VLAN) & vlan \\
    \hline
    Configuration Management & puppet \\
    \hline
    Monitoring & nagios \\
    \hline
    Virtual Private Network (VPN) & openvpn \\
    \hline
\end{tabular}
\end{center}
