\subsection{Services}
All existing services must be upgraded to accommodate the network upgrade. 
The scope of each service upgrade will vary based on the need, but each
service will be reimplemented to better fit within the post-upgrade network 
architecture. 
\subsubsection{Existing Services}
The following existing services are upgraded (in no particular order):
\newcolumntype{P}[1]{>{\centering\arraybackslash}p{#1}}
\begin{center}
\begin{tabular}{|P{7.3cm}|P{5.5cm}|}
    \hline
    \textbf{Service Description} & \textbf{Preferred Package/Application} \\
    \hline
    Network File System (NFS) & nfs-kernel-server, nfs-client \\
    \hline
    Webserver & apache2 \\
    \hline
    Database & mariadb \\
    \hline
    Email & ??? \\
    \hline
    Active Directory (AD) & openldap \\
    \hline
    Domain Name Server (DNS) & \textcolor{red}{-\$30,000} \\
    \hline
    Dynamic Host Configuration Protocol (DHCP) & dhcpd \\
    \hline
\end{tabular}
\end{center}

%%%%%%%%%%%%%%%%%%%%%%%%%%%%%%%%%%%%%%%%%%% start of database writeup %%%%%%%%%%%%%%%%%%%%%%%%%%%%%%%%%%%%%%%%%%%%%%%%%% 

\subsubsection{Database}
We are going with a MariaDB database for ACME as it widely used, easily 
maintainable, and secure. In order to set up the MariaDB database we would
 need to install it both the client and server packages.

\begin{itemize}

\item sudo apt-get install mariadb-server

\end{itemize}

\noindent After running the installation of MariaDB we then would need to set 
up our admin user so we are then able to populate the database with data that 
ACME CORP needs to be stored. 

\noindent By running:

\begin{itemize}
\item sudo mysql\_secure\_installation \\
\end{itemize}

\noindent We are able to do all of the following: 

\begin{itemize}
\item set root password
\item disable remote root login
\item remove test database
\item remove anonymous users and
\item reload privileges
\end{itemize}

\noindent After adding the data that needs to be stored we would then set up 
the user accounts and privileges for the HR department as they are the only 
department needing access to the database.

\begin{itemize}

\item MariaDB $>$ CREATE USER frankHR@'localhost' IDENTIFIED BY 'password'

\item MariaDB $>$ GRANT ALL PRIVILEGES on employees.* to frankHR@'\%';

\item MariaDB $>$ FLUSH PRIVILEGES; 

\end{itemize}

\noindent By the above commands we created a user account from Frank from HR 
and then have given Frank privileges to access the database 'employees'. 
Following this same style we would be able to add new tables and users and also
 give and take away privileges based on needs.

%%%%%%%%%%%%%%%%%%%%%%%%%%%%%%%%%%%%%%%%%%% end of database writeup %%%%%%%%%%%%%%%%%%%%%%%%%%%%%%%%%%%%%%%%%%%%%%%%%% 



%%%%%%%%%%%%%%%%%%%%%%%%%%%%%%%%%%%%%%%%%%% start of AD writeup %%%%%%%%%%%%%%%%%%%%%%%%%%%%%%%%%%%%%%%%%%%%%%%%%% 

\subsubsection{Active Directory}
We chose to use OpenLDAP as our active directory protocol.

\begin{itemize}
	
	\item sudo apt-get install slapd ldap-utils
	
\end{itemize}

\noindent This will prompt you administrator password for the administrator LDAP 
account.
After changing some recommended setting we then enter:

\begin{itemize}
	\item dpkg-reconfigure slapd
\end{itemize}

\noindent The above command reconfigures with the updated information we entered. After 
running this command we are then asked for numerous pieces of information

\begin{itemize}
	\item DNS domain name
	\item Organization name (ACME)
	\item LDAP admin password which we created earlier
	\item Selection of backend database. 
\end{itemize}

\noindent After the steps above we then test the OpenLDAP by running:

\begin{itemize}
	\item ldapsearch -x
\end{itemize}

\noindent If the ‘Success’ message outputs, then Congratulations! Our LDAP 
Server is working!! \\

\noindent Now to install the LDAP Server Administration portion. Since we will 
have a team of users that might not be great with computers, we will go with
the GUI tool. Which will help the manage and configure the LDAP server.
We install it with the following command:

\begin{itemize}
	\item sudo apt-get install phpldapadmin
\end{itemize}

\noindent We then have to set symbolic link for the phpldapadmin directory:

\begin{itemize}
	\item ln -s /usr/share/phpldapadmin/ /var/www/html/phpldapadmin
\end{itemize}

\noindent We then need to edit the config.php file for setting correct time 
zone:

\begin{itemize}
	\item vim /etc/phpldapadmin/config.php
\end{itemize}

\noindent We will look for a line: 
\begin{itemize}
	\item \begin{verbatim}
	$config->custom->appearance['timezone'] = ;
	\end{verbatim} 
\end{itemize} 

\noindent Change it to ACME Pennsylvania: 
\begin{itemize}
	\item \begin{verbatim}
	$$config->custom->appearance['timezone'] = 'US/Pennsylvania';
	\end{verbatim}
\end{itemize}

\noindent Lastely we need to find and replace the domain names with our own. 
Find "Define LDAP Servers" section with in config file and 
change the following lines:
\begin{verbatim}
	// Set your LDAP server name //
	$servers->setValue('server','name','Unixmen LDAP Server');
	
	// Set your LDAP server IP address // 
	$servers->setValue('server','host','192.168.1.103');
	
	// Set Server domain name //
	$servers->setValue('server','base',array('dc=unixmen,dc=local'));
	
	// Set Server domain name again//
	$servers->setValue('login','bind_id','cn=admin,dc=unixmen,dc=local');
\end{verbatim}

\noindent We need to restart the apache service using:
\begin{itemize}
	\item systemctl restart apache2
\end{itemize}

\noindent Now make sure port "80" and port "389" are open in the 
firewall/router config and we are finished.

%%%%%%%%%%%%%%%%%%%%%%%%%%%%%%%%%%%%%%%%%%% end of AD writeup %%%%%%%%%%%%%%%%%%%%%%%%%%%%%%%%%%%%%%%%%%%%%%%%%% 


\subsubsection{New Services}
The following new services are implemented on the new network (in no particular
order):
\newcolumntype{P}[1]{>{\centering\arraybackslash}p{#1}}
\begin{center}
\begin{tabular}{|P{7.3cm}|P{5.5cm}|}
    \hline
    \textbf{Service Description} & \textbf{Preferred Package/Application} \\
    \hline
    Virtual Local Area Network (VLAN) & vlan \\
    \hline
    Configuration Management & puppet \\
    \hline
    Monitoring & nagios \\
    \hline
    Virtual Private Network (VPN) & openvpn \\
    \hline
\end{tabular}
\end{center}
